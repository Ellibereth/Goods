\documentclass[12pt]{article}
\usepackage[english]{babel}
\usepackage{amsmath,amsthm,amsfonts,amssymb,epsfig}
\usepackage[left=.5in,top=.5in,right=.5in, bottom=1in]{geometry}
\usepackage{mathtools}
\usepackage{breqn}



\usepackage{titlesec}
\usepackage{verbatim}
\usepackage{amssymb}
\usepackage{amsmath}
\usepackage{amsthm}
\usepackage{mathrsfs}
\usepackage{enumitem}

\usepackage{graphicx}

%% commands for math note taking 
\newcommand{\defn} [0]  {\paragraph{DEFN : }}
\newcommand{\fact}[0] {\paragraph{FACT : }}
\newcommand{\prop} [0] {\paragraph{PROP  : }}
\newcommand{\thm}[0] {\paragraph{THM : }}
\newcommand{\cor}[0] {\paragraph{COROLLARY : }}
\newcommand{\pf}[0] {\paragraph*{PROOF :}}
\newcommand{\note}[0] {\paragraph*{NOTE :}}
\newcommand{\lemma}[0]{\paragraph*{LEMMA : }}
\DeclarePairedDelimiter{\abs}{\lvert}{\rvert}


\newcommand*{\backnotin}{\rotatebox[origin=c]{-180}{$\notin$}}%

\titleformat{\subsection}{\centering\large\bfseries}{\thesection}{1em}{}



\DeclareMathOperator{\tr}{tr}
\DeclareMathOperator{\im}{im}
\DeclareMathOperator{\Perm}{Perm}
\DeclareMathOperator{\lcm}{lcm}
\DeclareMathOperator{\Var}{Var}
\DeclareMathOperator{\Cov}{Cov}

\title{Amazon Notes}
\author{Darek Johnson}
\date{Feb 24, 2017}

\begin{document}

%%  eliminates page numbers
%%  \pagestyle{empty}

\maketitle

\section*{2/24/2017}
The goal of today was to think about and come up with a way to best store which items on Amazon and also how to get an initial database with the origins of a reasonable number of goods. \\\\
Each product on Amazon marketplace has a unique identifies called an ASIN. In our own database, we should use these as our own unique identifiers. If you consider any Amazon url it is of the following form\\\\
https://www.amazon.com/Russell-Athletic-T-Shirt-Heather-4X-Large/dp/B00719ZZWQ/... \\\\
Notice the .../dp/B00719ZZWQ/... the B00719ZZWQ is the ASIN for this product. It always comes after the /dp/ in the URL. This is at least true for the 30 random goods I checked today. In the past dp has been changed with other things like /product or /gp, but as of today it is /dp. Also because of the uniqueness of ASIN the following url is equivalent to the one above amazon.com/dp/B00719ZZWQ from a product standpoint (only minor differences). \\\\
In other news I started an Amazon Associates account that is free for 180 days, and this gives us access to Amazon Product API. This API is compatible with PHP, Java, ROR directly, but there are third party packages for Python. Java is probably best since third party usually just has worse documentation. If you want to see an example of what kind of information is provided from Amazon API, see test.xml and test2.xml in this folder. One important thing to notice is that the "origin" field that sometimes occurs in the product description area, is not returned in the Amazon API. This makes things a little more difficult, but things of note are the following

\begin{itemize}
	\item Manufacturer 
	\item Barcode (UPC, EAN)
	\item ASIN ID of a parent good (related good)
	\item Brand 
	\item Model Number
\end{itemize}


To get our database started we can get a list of manufacturers we know are in the USA and use the API to find some of these and add the ASIN to the database. Next things to do are to start writing our own API in java to do these searches and set up our database manager in java. I'll ask Eli what database we want to store these things in.

\section*{2/28/2017}
Since last we have recovered all the XMLs for Amazons subcategories. This is the hierarchy for Amazon products. This has given us a lot of the frequently searched keywords. What I have done so far is stored these XML files, then finding a node with no children, use the name of that node and append it with the root ancestor, for example Home \& Goods Chairs, uses this as a keyword search Amazon, then from the 20 pages of results, HTML scrapes to get each ASINs from those pages. We store the ASINs in asin_list.csv. Regardless of what we do with Amazon, getting a list of ASINs will be very important. \\ \\
Things to do soon,

\begin{itemize}
	\item Set up multiple rotating proxies to scrape all of these ASINs without getting caught
	\item Use ASIN to get the following information: price + company + country of origin + name of product + classification of product
	\item Amazon API only allows 2000 requests an hour, so how do we get around this? What should we use Amazon API for and how. If this is limiting, what can we scrape instead.
	\item Rewrite parse.py to run requests concurrently to optimize using grequests library
	\item Use try and catch statements around anything HTTP requests or things that could break. Log what's broken somewhere not in console
\end{itemize}

Overall structure, we should have the following python files

\begin{itemize}
	\item amazon.py - Will do all the calls to the Amazon Associates API
	\item parse.py - Will do all the html scraping from Amazon's Website 
	\item data_manager.py - Will have the more complicated functionality for writing to and from CSVs and the database. Simple stuff can be within other classes
	\item main.py - This isn't made yet, but will eventually just import the other classes and use them. Eventually this should be run on our server every night or something to update our databse
\end{itemize}

The data we have to work with right now is a database of 20k ASIN numbers, and the entire Amazon Organizational Tree downloaded in the form of XML files. Next steps regarding ASIN is to get the 5 previously mentioned data points for each ASIN (maybe some more too)




\end{document}





















